\documentclass[a4paper,12pt]{article}
\usepackage[utf8]{inputenc}
\usepackage{amsmath}
\usepackage{listings}
\usepackage{color}

\title{Documentação do Código: Lista Encadeada Genérica em C++}
\date{\today}

\begin{document}

\maketitle

\section{Introdução}

Este documento descreve a implementação de uma lista encadeada genérica em C++ utilizando templates. A lista encadeada é uma estrutura de dados linear onde os elementos são armazenados em nós (objetos da classe \texttt{No}), e cada nó aponta para o próximo nó na sequência.

\section{Classe \texttt{No}}

A classe \texttt{No} representa um único nó na lista encadeada. Cada nó armazena um elemento do tipo genérico \texttt{K} e um ponteiro para o próximo nó na lista.

\begin{lstlisting}[language=C++, caption=Classe No]
template <typename K>
class No {
    public:
        K element;
        No* next;

        No(K element) {
            this->element = element;
            this->next = nullptr;
        }
};
\end{lstlisting}

\begin{itemize}
    \item \textbf{Membros da Classe:}
    \begin{itemize}
        \item \texttt{element}: Armazena o dado do tipo \texttt{K}.
        \item \texttt{next}: Ponteiro para o próximo nó na lista.
    \end{itemize}
    \item \textbf{Construtor:}
    \begin{itemize}
        \item Recebe um elemento do tipo \texttt{K} e o armazena no nó, enquanto o ponteiro \texttt{next} é inicializado como \texttt{nullptr}.
    \end{itemize}
\end{itemize}

\section{Classe \texttt{LinkedList}}

A classe \texttt{LinkedList} implementa a lista encadeada. Ela gerencia a adição e remoção de elementos na lista e mantém o controle do comprimento da lista.

\begin{lstlisting}[language=C++, caption=Classe LinkedList]
template <typename T>
class LinkedList {
    public:
        No<T> *first, *last;
        int length;

        LinkedList() {
            this->first = nullptr;
            this->last = nullptr;
            this->length = 0;
        }
        
        // Métodos de adição e remoção de elementos
        void push_end(T element);
        void push(T element, int pos);
        void push_start(T element);
        No<T>* pop_end();
        No<T>* pop(int pos);
        No<T>* pop_start();
        int search(T element);
        No<T>* get(int pos);
        void show();
        ~LinkedList();
};
\end{lstlisting}

\subsection{Membros da Classe}

\begin{itemize}
    \item \texttt{first}: Ponteiro para o primeiro nó da lista.
    \item \texttt{last}: Ponteiro para o último nó da lista.
    \item \texttt{length}: Contador do número de elementos na lista.
\end{itemize}

\subsection{Construtor}

O construtor inicializa uma lista vazia, com \texttt{first} e \texttt{last} como \texttt{nullptr} e \texttt{length} como 0.

\subsection{Métodos Principais}

\begin{itemize}
    \item \textbf{\texttt{push\_end(T element)}}: Adiciona um elemento ao final da lista.
    \item \textbf{\texttt{push(T element, int pos)}}: Insere um elemento em uma posição específica.
    \item \textbf{\texttt{push\_start(T element)}}: Adiciona um elemento no início da lista.
    \item \textbf{\texttt{pop\_end()}}: Remove o último elemento da lista e retorna o nó removido.
    \item \textbf{\texttt{pop(int pos)}}: Remove e retorna o nó na posição especificada.
    \item \textbf{\texttt{pop\_start()}}: Remove e retorna o primeiro nó da lista.
    \item \textbf{\texttt{search(T element)}}: Procura por um elemento na lista e retorna sua posição. Se o elemento não for encontrado, retorna -1.
    \item \textbf{\texttt{get(int pos)}}: Retorna o nó na posição especificada.
    \item \textbf{\texttt{show()}}: Imprime todos os elementos da lista na ordem em que estão armazenados.
    \item \textbf{Destrutor \texttt{\~LinkedList()}}: Libera a memória de todos os nós ao destruir a lista, evitando vazamentos de memória.
\end{itemize}

\section{Função \texttt{main}}

A função \texttt{main} cria uma lista encadeada de inteiros e interage com o usuário para adicionar e remover elementos, exibindo a lista após cada operação.

\begin{lstlisting}[language=C++, caption=Função main]
int main(int argc, char const *argv[])
{
    LinkedList<int> lista;
    int n, k, elemento;

    cout << "Quantos elementos deseja incerir na lista? ";
    cin >> n;

    for (int i = 0; i < n; i++) {
        cout << "Digite o elemento " << i + 1 << ": ";
        cin >> elemento;
        lista.push_end(elemento);
    }

    lista.show();

    cout << "Quantos elementos deseja remover da lista? ";
    cin >> k;

    for (int i = 0; i < k; i++) {
        cout << endl << endl;
        lista.show();
        cout << "Posicao do elemento que deseja remover: " << endl;
        int temp = 0;
        cin >> temp;
        try {
            No<int>* tempNo = lista.pop(temp);
            cout << "Elemento removido: " << tempNo->element << endl;
            delete tempNo;
        } catch (runtime_error& e) {
            cout << e.what() << endl;
        }
    }

    lista.show();
    cout << "Posicao do elemento que deseja printar: " << endl;
    cin >> k;
    No<int>* tempNo = lista.get(k);
    cout << tempNo->element << endl;
    delete tempNo;

    return 0;
}
\end{lstlisting}

\subsection{Interação com o Usuário}

O usuário é solicitado a inserir um número de elementos na lista e, posteriormente, remover elementos de posições específicas. Além disso, o usuário pode consultar e imprimir um elemento na posição desejada.

\subsection{Erros de Execução}

Exceções são lançadas se o usuário tentar remover um elemento de uma lista vazia ou de uma posição fora do intervalo válido. Além disso, o método \texttt{search} retorna -1 se o elemento não for encontrado.

\section{Resumo}

Este código implementa uma lista encadeada genérica com operações básicas como inserção, remoção e busca de elementos em posições específicas. Ele utiliza templates para permitir que a lista trabalhe com qualquer tipo de dado. Além disso, a função \texttt{main} demonstra o uso prático da lista, interagindo com o usuário para adicionar, remover e consultar elementos.

\end{document}
