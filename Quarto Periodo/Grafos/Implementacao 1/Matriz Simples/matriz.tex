\documentclass{article}
\usepackage[utf8]{inputenc}
\usepackage[brazil]{babel}

\title{Documentação - Implementação de uma Matriz não flexível}
\author{}
\date{}

\begin{document}

\maketitle

\section{Introdução}

Esta documentação descreve a implementação de uma classe \texttt{Matriz} em C++ que representa uma matriz não flexível. A classe utiliza um vetor de vetores (\texttt{std::vector<std::vector<int>>}) para armazenar os elementos da matriz e fornece métodos para inserção, verificação de existência, remoção e impressão dos elementos.

\section{Estrutura da Classe \texttt{Matriz}}

A classe \texttt{Matriz} possui os seguintes membros:

\subsection{Membros Privados}

\begin{itemize}
    \item \texttt{std::vector<std::vector<int>> data}: Um vetor de vetores que armazena os elementos da matriz.
\end{itemize}

\subsection{Membros Públicos}

\begin{itemize}
    \item \texttt{void insert(int row, int col, int value)}: Insere um valor \texttt{value} na linha \texttt{row} e coluna \texttt{col} da matriz. Se a matriz não for grande o suficiente para acomodar o novo elemento, ela é redimensionada automaticamente.
    \item \texttt{bool contains(int value)}: Verifica se um valor \texttt{value} está presente na matriz. Retorna \texttt{true} se o valor estiver presente, caso contrário, retorna \texttt{false}.
    \item \texttt{void remove(int value, int k)}: Remove as primeiras \texttt{k} ocorrências do valor \texttt{value} da matriz.
    \item \texttt{void print()}: Imprime a matriz no console.
\end{itemize}

\section{Exemplo de Uso}

O código a seguir demonstra como usar a classe \texttt{Matriz}:

\begin{verbatim}
#include <iostream>

int main() {
    Matriz matriz;

    // Inserindo elementos na matriz
    matriz.insert(0, 0, 1);
    matriz.insert(0, 1, 2);
    matriz.insert(1, 0, 3);
    matriz.insert(1, 1, 4);

    // Imprimindo a matriz
    std::cout << "Matriz:\n";
    matriz.print();

    // Verificando se o valor 3 está presente na matriz
    if (matriz.contains(3)) {
        std::cout << "O valor 3 está na matriz\n";
    } else {
        std::cout << "O valor 3 não está na matriz\n";
    }

    // Removendo o valor 2 da matriz
    matriz.remove(2, 1);

    // Imprimindo a matriz após a remoção
    std::cout << "Matriz após a remoção:\n";
    matriz.print();

    return 0;
}
\end{verbatim}

\section{Observações}

\begin{itemize}
    \item A matriz é redimensionada automaticamente quando um elemento é inserido em uma linha ou coluna inexistente. Os novos elementos são inicializados com o valor 0.
    \item A remoção de elementos é feita em ordem de ocorrência na matriz, linha por linha.
\end{itemize}

\end{document}