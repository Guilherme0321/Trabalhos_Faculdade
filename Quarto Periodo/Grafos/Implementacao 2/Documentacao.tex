\documentclass[a4paper,12pt]{article}
\usepackage{amsmath}
\usepackage{listings}
\usepackage{graphicx}
\usepackage{hyperref}
\usepackage{amsfonts}
\usepackage{amssymb}
\usepackage[utf8]{inputenc}

\title{Documentação: Gerador de Subgrafos de um Grafo Completo}
\author{Ana Fernanda Souza Cancado \\ Gabriel Praes Bernardes Nunes\\ Guilherme Otávio de Oliveira \\ Júlia Pinheiro Roque}
\date{\today}

\begin{document}

\maketitle

\section{Introdução}

Este documento descreve a implementação de um programa em C++ que gera todos os subgrafos possíveis de um grafo completo com $n$ vértices. O programa também conta o número total de subgrafos gerados. O valor de $n$ é fornecido pelo usuário.

\section{Descrição da Classe \texttt{Graph}}

A classe \texttt{Graph} é responsável por representar o grafo e por realizar as operações de geração de subgrafos. Abaixo está a descrição dos principais métodos e atributos da classe.

\subsection{Atributos Privados}
\begin{itemize}
    \item \texttt{int V}: Representa o número de vértices do grafo.
    \item \texttt{std::vector<std::list<int>> adj}: Vetor de listas que representa a lista de adjacência do grafo.
\end{itemize}

\subsection{Atributo Público}
\begin{itemize}
    \item \texttt{int numeroDeSubgrafos = 0}: Armazena o número total de subgrafos gerados.
\end{itemize}

\subsection{Métodos Públicos}
\begin{itemize}
    \item \texttt{Graph(int V)}: Construtor que inicializa o grafo com $V$ vértices.
    \item \texttt{void addEdge(int v, int w)}: Adiciona uma aresta entre os vértices $v$ e $w$.
    \item \texttt{void createCompleteGraph()}: Cria um grafo completo, onde todos os pares de vértices estão conectados.
    \item \texttt{void printVertices()}: Imprime todos os vértices do grafo.
    \item \texttt{unsigned long long factorial(int num)}: Calcula o fatorial de um número.
    \item \texttt{unsigned long long combination(int n, int r)}: Calcula o número de combinações possíveis de $r$ elementos entre $n$.
    \item \texttt{void printAllSubgraphs()}: Gera e imprime todos os subgrafos possíveis do grafo, bem como suas combinações de arestas.
\end{itemize}

\section{Funcionamento do Programa}

O programa começa solicitando ao usuário o número de vértices $n$. Em seguida, ele cria um grafo completo com $n$ vértices, onde todos os vértices estão conectados entre si.

O método \texttt{printAllSubgraphs()} é responsável por gerar e imprimir todos os subgrafos possíveis. Ele faz isso iterando sobre todas as combinações de vértices e gerando todas as possíveis combinações de arestas para cada conjunto de vértices.

O número total de subgrafos gerados é armazenado no atributo \texttt{numeroDeSubgrafos}, que é impresso ao final da execução do programa.

\section{Exemplo de Saída}

Suponha que o usuário insira $n = 3$. A saída do programa será semelhante ao seguinte:

\begin{verbatim}
n (vertices): 3
[1, 2, 3]
Subgrafos com r = 1:
[1] {}
[2] {}
[3] {}
Subgrafos com r = 2:
[1, 2] {}
[1, 2] {(1, 2)}
[1, 3] {}
[1, 3] {(1, 3)}
[2, 3] {}
[2, 3] {(2, 3)}
Subgrafos com r = 3:
[1, 2, 3] {}
[1, 2, 3] {(1, 2)}
[1, 2, 3] {(1, 3)}
[1, 2, 3] {(2, 3)}
[1, 2, 3] {(1, 2), (1, 3)}
[1, 2, 3] {(1, 2), (2, 3)}
[1, 2, 3] {(1, 3), (2, 3)}
[1, 2, 3] {(1, 2), (1, 3), (2, 3)}
Numero de subgrafos: 17
\end{verbatim}

\section{Conclusão}

O programa documentado aqui é uma ferramenta útil para entender a estrutura de subgrafos em grafos completos. Ele gera todos os subgrafos possíveis e conta quantos subgrafos foram gerados, oferecendo uma visão detalhada das possíveis combinações de vértices e arestas.

\end{document}
